%------------------------------------------------------------------------------
%---------------- Docmentenklasse, Layout und Ränder: -------------------------
%------------------------------------------------------------------------------
\RequirePackage{fixltx2e}
\documentclass[
        paper=a4,               % Papierformat DIN A4
        BCOR=12mm,              % 12mm Binderandkorrektur
        %parskip=false,           % Absätze als halbe Leerzeile
        headsepline,            % Linie zwischen Kopfzeile und Dokument
        cleardoublepage=plain,  % Keine Kopf/Fußzeile auf Leerseiten
        bibliography=totoc,     % Bibliographie als nicht-nummeriertes 
                                % Kapitel im Inhaltsverzeichnis
        open=right,             % Kapitel beginnen immer aufrechten Seiten
        open=any,               % Kapitel dürfen auf beiden Seiten beginnen
        numbers=noenddot,       % Keine Punkte nach Abbildungsnummern
        captions=tableheading,  % Spacing für Captions über Tabellen an
        titlepage=firstiscover, % Titelseite ist Deckblatt, symmetrische Ränder
        %headings=normal         % kleinere Überschriften
    ]
    {scrbook}


% Patch für LaTeX laden:

% Warnung, falls erneut kompilirt werden muss:
\usepackage[aux]{rerunfilecheck}

% Beschränkung auf chapter und section im Inhaltsverzeichnis:            
\setcounter{tocdepth}{1}

%------------------------------------------------------------------------------
%------------------------------ Sprache und Schrift: --------------------------
%------------------------------------------------------------------------------

% Deutsche Spracheinstellungen
\usepackage{polyglossia}
\setdefaultlanguage{german}

% stellt den \enquote{} Befehl
\usepackage[autostyle]{csquotes}

% verbessert das allgemeine Schriftbild:
\usepackage{microtype}

\usepackage{fontspec}
\defaultfontfeatures{Ligatures=TeX}


\hyphenation{Si-mu-la-tions-ket-te}


%------------------------------------------------------------------------------
%-------------------------   Kopf/Fußzeile/Farben------------------------------
%------------------------------------------------------------------------------


\usepackage{xcolor}
\xdefinecolor{tugreen}{RGB}{128, 186, 38}

\usepackage{scrlayer-scrpage}
\pagestyle{scrheadings}

\addtokomafont{title}{\color{tugreen}}          % Titel auf der Titelseite in TU-Grün
\addtokomafont{sectioning}{\color{tugreen}}     % Kapitel-Überschriften in TU-Grün
\addtokomafont{pagenumber}{\color{tugreen}}     % Seitenzahl in TU-Grün
\addtokomafont{chapterentry}{\color{tugreen}}   % Inhaltsverziechnis
\setheadsepline{0.75pt}[\color{tugreen}]        % Linie unter Kopzzeile in TU-Grün


%------------------------------------------------------------------------------
%------------------------ Für die Matheumgebung--------------------------------
%------------------------------------------------------------------------------

\usepackage{amsmath}
\usepackage{amssymb}
\usepackage{mathtools}

% Enable Unicode-Math and follow the ISO-Standards for typesetting math
\usepackage[
  math-style=ISO,
  bold-style=ISO,
  sans-style=italic,
  nabla=upright,
  partial=upright,
]{unicode-math}
\setmathfont{Latin Modern Math}

% schöne Brüche im Text mit \sfrac{}{}
\usepackage{xfrac}  


%Gleichungsnummern Kapitel.Unterkapitel.Gleichung
\renewcommand{\theequation}{\thechapter{}.\arabic{equation}}
\numberwithin{equation}{chapter}


%------------------------------------------------------------------------------
%---------------------------- Einheiten ---------------------------------------
%------------------------------------------------------------------------------

\usepackage[
  locale=DE,
  separate-uncertainty=true,
  per-mode=symbol-or-fraction,
]{siunitx}
\sisetup{math-micro=\text{µ},text-micro=µ}

%------------------------------------------------------------------------------
%------------------------------ Tabellen: -------------------------------------
%------------------------------------------------------------------------------

\usepackage{booktabs}       % stellt \toprule, \midrule, \bottomrule
\usepackage{threeparttable} % für komplexere Tabellen

%------------------------------------------------------------------------------
%------------------------------ Grafiken: -------------------------------------
%------------------------------------------------------------------------------

\usepackage[]{graphicx}

\usepackage{grffile}

\usepackage{scrhack}
\usepackage{float}
\floatplacement{figure}{htbp}
\floatplacement{table}{htbp}
\usepackage[section]{placeins}

\usepackage{caption}
\DeclareCaptionFont{tugreen}{\color{tugreen}}
\captionsetup{%
            labelfont={bf,tugreen},     % Label fett und in TU-Grün
            format=plain,               % Caption-Text steht auch unter "Tabelle x"
            width=0.9\textwidth         % Bereich für Caption schmaler als für Text
           }

\usepackage{subcaption}   % for subfigures


%------------------------------------------------------------------------------
%------------------------------ Bibliographie ---------------------------------
%------------------------------------------------------------------------------

\usepackage[backend=biber]{biblatex}    % Biblatex mit biber
    \addbibresource{references.bib}     % die Bibliographie einbinden

\DefineBibliographyStrings{ngerman}{andothers = {{et\,al\adddot}}}


%------------------------------------------------------------------------------
%------------------------------ Sonstiges: ------------------------------------
%------------------------------------------------------------------------------


\usepackage{eurosym}
\usepackage[pdfusetitle,unicode, linkbordercolor=tugreen]{hyperref}
\usepackage{bookmark}
\usepackage[shortcuts]{extdash}


