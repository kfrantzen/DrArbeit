\chapter{Zusammenfassung}
\label{chapter:Ergebnisse}
Im Rahmen dieser Arbeit wurde zunächst eine Einführung in die Astroteilchenphysik gegeben. 
Die kosmische Strahlung wurde beschrieben und Quelle dieser Strahlung vorgestellt.
Anschließend wurden mögliche Beschleunigungsmechanismen dargestellt.
Die Überleitung zum Gebiet der Gammaastronomie führt in das eigentliche Forschungsthema dieser Arbeit ein.
Dieses Thema umfasst die Analyse des Aktiven Galaktischen Kerns (AGN) Markarian~421, welcher nachfolgend vorgestellt wird.

Ein essenzieller Grundstein für diese Analyse sind die speziell angepassten Monte Carlo-Simulationsdaten.
Diese wurden im Zeitraum der Doktorarbeit in Dortmund für die gesamte MAGIC-Kollaboration produziert und an aktuelle Hardwareveränderungen des Teleskops angepasst.
Beginnend mit der Luftschauersimulation mit \textit{CORSIKA} werden nacheinander die Reflektor-Simulation und die Kamera- und Elektronik-Simulation vorgestellt.
Diese Programme benötigen spezielle Eingabeparameter, welche ebenfalls beschrieben werden.
Diesen reinen Simulationsprogrammen folgen die Kalibrationsprogramme. 
Nach dem Durchlaufen durch die Simulations- und die Kalibrationsprogramme liegen die Monte-Carlo-Simulationsdaten im gleichen Format vor, wie die realen Daten, welche der Kollaboration zur Verfügung stehen.
Um eine schnelle Produktion von ausreichend MC-Daten zu gewährleisten wird die an der TU Dortmund implementierte automatische Produktionskette genutzt.
Diese aus Skripten und einer Datenbank bestehende Kette wurde im Verlaufe der Doktorarbeit betreut und die Skripte an aktuelle Hardwareveränderungen angepasst.

Im Analyse-Teil dieser Arbeit wird die Bedeutung der MC-Simulationsdaten in den einzelnen Programmen herausgestellt.
Es wird zunächst die Methode des Random Forests in der Gamma-Hadron-Separation und in der Rekonstruktion der Quellposition vorgestellt.
Danach wird die Energierekonstruktion mit Hilfe von Look-Up-Tabellen erklärt.
Abschließend wird noch die Methode der Entfaltung kurz beschrieben.
Nachdem diese Methoden der Analyse vorgestellt wurden, erfolgt die Analyse der Daten der AGN Mrk~421 von 2012.
Diese Daten wurden während des Teleskop-Upgrades genommen und weisen deswegen eine größe Lücke in der Datennahme zwischen Juni und Dezember auf. 
Des Weiteren mussten die Daten aufgrund verschiedener Hardwareveränderungen und Spiegeleigenschaften in vier separate Datensätze getrennt werden, die jeweils vier separate MC-Datensätze erfordern.
Ein Datensatz besteht aus Mono-Beobachtung, da die alte Kamera von MAGIC-I ihren Geist aufgab und keine stereoskopische Beobachtung mehr möglich war, was wiederum zu einem speziell angepassten MC-Datensatz führte.
Aus der Analyse dieser einzelnen Datensätze folgt, dass die Quelle Mrk~421 2012 keine Flares aufweist.
Ihr Fluss liegt zwischen 24\% und 45\% des Flusses von Crab und somit auf einem für diese Quelle konstant niedrigem Niveau.

Anschließend an die Analyse der Daten mit MAGIC wurden die Daten einer Multiwellenlängenkampagne, die 2012 ebenfalls stattfand, ausgewertet.
Diese enthält Beobachtungen unterschiedlicher Instrumente wie bodengebundener Teleskope oder Satellitenexperimenten.
Die Lichtkurven der einzelnen Instrumente wurden in einem Diagramm dargestellt und ihre Variabilität untersucht.
Diese Untersuchung hat ergeben, dass die Quelle während der Multiwellenlängenkampagne vor allem variable Gamma- und Röntgenstrahlung emittiert.
Mit steigender Frequenz steigt auch die Variabilität.
Diese Ergebnisse sind vergleichbar zu den Ergebnissen, die eine MWL-Studie 2009 geliefert hat, wobei diese Studie die höchste Variabilität für Röntgenstrahlung gefunden hat.
Die hohe Variabilität für Röntgenstrahlung lässt Hinweise auf das SSC-Modell zu.
Allerding sind hadronische Modelle auch noch nicht ausgeschlossen, da eine erfolgreiche Beobachtung dieser Quelle mit Neutrinoteleskopen noch nicht gelang.

In Zukunft könnten solche Neutrinoteleskope Aufschlüsse über die Emissionsmodelle in extragalaktischen Quellen geben, sofern sie Neutrinos detektieren.
Dann könnte die Physik im Quelltyp der AGN besser verstanden werden.

Die essenzielle Wichtigkeit von MC-Simulationen in der Analyse der Daten erfordert eine zeitnahe Simulation.
Mit Hilfe einer weiterhin schnellen Produktion, die optimal an die Hardware angepasst ist, können gute Analyseergebnisse erzielt werden.
Technisch gesehen, könnte dies durch eine Aktualisierung der MC-Kette unter Benutzung einer aktuellen Programmiersprache und den Einbau von Fehlermeldungen geschehen.
Des Weiteren wäre es von großem Vorteil, die Laufzeit von \textit{CORSIKA} zu verringern, da dieses Programm mit Abstand die llängste Laufzeit benötigt.