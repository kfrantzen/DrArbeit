\chapter{MrK421 Analyse}
In diesem Kapitel wird die Analyse der Daten beschrieben, wobei für jede Datenepoche sowohl Lichtkurve als auch Spektrum gezeigt werden.
Dabei wird die Analyse des zweiten Teils der Daten exemplarisch für die Stereo-Analyse erklärt ABSCHNITT BLABLA, während die anderen Zeitabschnitte des Jahres mit stereoskopischer Beobachtung ABSCHNITT BLA UND BLA analog ausgewertet werden.
Auf die Mono-Datennanalyse wird in section BLABLA danach eingegangen.
Zusammenfassend wird noch eine Lichtkurve aller Daten gezeigt.


\section{Überblick über die Daten}
Die Daten, die mir zur Verfügung standen, sind Daten der Quelle Mrk421, die 2012 genommen wurden.
Ein Nachteil dieser Daten ist, dass in diesem Jahr einige Hardware-Veränderungen statt fanden und die Kamera von MAGIC 1 kaputt gegangen ist wegen Sabrina, die aber Pause hatte.
Aufgrund dieser Tatsache müssen vier verschiedene MC-Sets in der Analyse verwendet werden.


\section{Teil2.2}
Anhand der genommenen Daten zwischen dem 18.3. und dem 27.4. wird nun die Analyse erklärt.
Die anderen Abschnitte mit stereoskopisch genommenen Daten werden analog dazu analysiert. 


\subsection{Data-/Off-Data-/MC-/Crab-Selection und Qualitätsmerkmale}


\subsection{Lichtkurve}


\subsection{Spektrum}
Crabspektrum
richtiges mrk421


\section{Teil2.1}

\section{}
